\documentclass[11pt]{article}
\usepackage{multicol}
\usepackage{ifthen}
%\usepackage{multitoc}
%\usepackage{german}
%\usepackage{bibgerm}
\usepackage{amsmath}
\usepackage{amsfonts}
\usepackage{color}
\usepackage{hyperref}
\usepackage[dvips]{epsfig}
\usepackage[dvips]{graphicx}
\usepackage[a4paper,body={148mm,240mm,nohead}]{geometry}
\usepackage[ansinew]{inputenc}
\usepackage{listings}
\lstset{language=C++, indent=20pt, basicstyle=\ttfamily,
  stringstyle=\ttfamily, commentstyle=\it, extendedchars=true}

\newif\ifpdf
\ifx\pdfoutput\undefined
\pdffalse % we are not running PDFLaTeX
\else
\pdfoutput=1 % we are running PDFLaTeX
\pdftrue
\fi

\ifpdf
\usepackage[pdftex]{graphicx}
\else
\usepackage{graphicx}
\fi

\ifpdf
\DeclareGraphicsExtensions{.pdf, .jpg, .tif}
\else
\DeclareGraphicsExtensions{.eps, .jpg}
\fi

\newcommand{\C}{\mathbb{C}}
\newcommand{\R}{\mathbb{R}}
\newcommand{\N}{\mathbb{N}}
\newcommand{\Z}{\mathbb{Z}}
\newcommand{\Q}{\mathbb{Q}}
\newcommand{\K}{\mathbb{K}}


\title{Iterative Solver Template Library\thanks{Part of the
    Distributed and Unified Numerics Environment (DUNE) which is
    available from the site
    \texttt{http://www.dune.uni-hd.de/}}}

\author{%
Peter Bastian\\
Interdisziplin�res Zentrum f�r Wissenschaftliches Rechnen,\\
Universit�t Heidelberg, Im Neuenheimer Feld 368, D-69120 Heidelberg, \\
email: \texttt{Peter.Bastian@iwr.uni-heidelberg.de}}

\date{\today}

\begin{document}

\maketitle

\begin{abstract}
This document describes the rationale behind and use of the Iterative
Solver Template Library (ISTL) which provides a set of C++
templates to represent vectors, (sparse) matrices and some generic
algorithms based on these. The most prominent features of the
matrix/vector classes is that they support a recursive block structure
in a bottom up way. The classes can be used, e.~g., to efficiently implement
block preconditioners for $hp$-finite elements.
\end{abstract}

\begin{multicols}{2}
{\small\tableofcontents}
\end{multicols}

\section{Introduction}

The numerical solution of partial differential equations (PDEs) frequently requires the
solution of large and sparse linear systems. Naturally,
there are many libraries available on the internet for doing sparse matrix/vector
computations. A comprehensive overview is given in \cite{LALinks}. 

The
widely availably Basic Linear Algebra Subprograms (BLAS) standard has
been extended to cover als sparse matrices \cite{BLASTForum}. BLAS
divides the available functions into level 1 (vector operations),
level 2 (vector/matrix operations) and level 3 (matrix/matrix
operations). BLAS for sparse matrices contains only level 1 and 2
functionality and is quite different to the standard for dense
matrices. The standard uses procedural programming style and offers
only a FORTRAN and C interface. As a consequence, the interface is
``coarse grained'', meaning that ``small'' functions such as access to
individual matrix elements is relatively slow. 

Generic programming techniqes in C++ offer the possibility to combine
flexibility and reuse (``efficiency of the programmer'') with fast
execution (``efficieny of the program'') as has been demonstrated with
the Standard Template Library (STL), \cite{Stroustrup} or the Blitz++
library for multidimensional arrays \cite{Blitz}. A variety of
template programming techniques such as traits, template metaprograms,
expression templates or the Barton-Nackman trick are used in the
implementations, see \cite{BN,Veldhui99} for an introduction.
Application of these ideas to matrix/vector operations is available
with the Matrix Template Library (MTL), \cite{MTL,MTL_SciTools98}. The Iterative
Template Library (ITL), \cite{ITL}, implements iterative solvers for
linear systems (mostly Krylov subspace methods) in a generic way based
on MTL. The Distributed and Unified Numerics Environment (DUNE),
\cite{Dune,DuneWeb}, applies the STL ideas to finite element
computations. 

Why bother with yet another OO-implementation of (sparse) linear
algebra when libraries, most notably the MTL, are available? The most
important reason is that the functionality in existing libraries has
not been designed specifically with advanced finite element methods in
mind. Sparse matrices in finite element computations have a lot of
structure. Here are some examples:
\begin{itemize}
\item Certain discretizations for systems of PDEs or higher order
  methods result in matrices where individual entries are replaced by
  small blocks, say of size $2\time 2$ or $4\times 4$. Straightforward
  iterative methods solve these small blocks exactly, see
  e.~g.~\cite{BH99}.
\item Equation-wise ordering for systems results in matrices having an
  $n\times n$ block structure where $n$ corresponds to the number of
  variables in the PDE and the blocks themselves are large. As an
  example we mention the Stokes system. Iterative solvers such as the
  SIMPLE or Uzawa algorithm use this structure.
\item Other structures that can be exploited are the level structure
  arising from hierarchic meshes, a p-hierarchic
  structure (e.~g.~decomposition in linear and quadratic part),
  geometric structure from decomposition in subdomains or topological
  structure where unknowns are associated with nodes, edges, faces or
  elements of a mesh.
\end{itemize}
It is very important to note that this structure is typically known at
compile-time and this knowledge should be exploited to produce
efficient code. Moreover, block structuredness is recursive,
i.~e.~blocks are build from blocks which are themselves build from
blocks. 

The Matrix Template Library also offers the possibility to partition a
matrix into blocks. However, their concept is top-down, i.~e.~an
already existing matrix is enriched by additional information to
implement the block structure. This is done at run-time and might thus
be less efficient and requires additional memory. In contrast the
bottom-up composition of block matrices from blocks can save memory.
We would like to stress that the library to be presented in this paper
is not nearly as broad in scope as the MTL.

\section{Vectors}

\subsection{Vector spaces}

In mathemetics vectors are elements of a vector space. A vector space
$V(\K)$, defined over a field $\K$, is a set of elements with two
operations: (i) vector space addition $+ : V\times V \to V$ and (ii) scalar
multiplication $* : \K\times V \to V$. These operations obey certain formal
rules, see your favourite textbook on linear algebra. In addition a
vector space may be normed, i.~e.~there is a function (obeying certain
rules) $\|.\| : V \to \R$ which measures distance in the vector
space. Even more specialized vector spaces have a scalar product which
is a function $\cdot : V\times V \to \K$. 

How do you construct a vector space? The easiest way is to take a
field, such as $\K=\R$ or $\K=\C$ and take a tensor product:
\begin{equation*}
V = \K^n = \underbrace{\K\times\K\times\ldots\times\K}_{\text{$n$ times}}.
\end{equation*} 
$n\in\N$ is called the dimension of the vector space. There are also
infinite-dimensional vector spaceswhich are, however, not of interest
in the context here. The idea of tensor products can be generalized.
If we have vector spaces $V_1(\K),\ldots,V_n(\K)$ we can construct a
new vector space by setting
\begin{equation*}
V(\K) = V_1\times V_2 \times \ldots \times V_n.
\end{equation*} 
The $V_i$ can be \textit{any} vector space over the field $\K$. The
dimension of $V$ is the sum of the dimensions of the $V_i$. For a
a mathematician every finite-dimensional vector space is isomorphic to
the $\R^k$ for an appropriate $k$ but in our applications it is
important to know the difference between $(\R^2)^7$ and
$\R^{14}$. Having these remarks about vector spaces in mind we can now
turn to the class design.

\subsection{Vector classes}

ISTL provides the following classes to make up vector spaces:

\paragraph{FieldVector}

The \lstinline!template<class K, int n> FieldVector<K,n>! class
template is used to represent a vector space
$V=\K^n$ where the field is given by the type
\lstinline!K!.  \lstinline!K! may be \lstinline!double!, \lstinline!float!,
\lstinline!complex<double>! or any other numeric type.
The dimension given by the template parameter
\lstinline!n! is assumed to be small. Members of this class are
implemented with template metaprograms to avoid tiny loops.
Example: Use \lstinline!FieldVector<double,2>! to define vectors with
a fixed dimension 2.

\paragraph{BlockVector}

The \lstinline!template<class B> BlockVector<B>! class template builds
a vector space $V=B^n$ where the ``block type'' $B$ is given by the
template parameter \lstinline!B!. \lstinline!B! may be any other class
implementing the vector interface. The number of blocks $n$ is given
at run-time. Example: 
\begin{lstlisting}{}
BlockVector<FieldVector<double,2> >
\end{lstlisting}
can be used to define vectors of variable size where each block in turn
consists of two \lstinline!double! values.

\paragraph{VariableBlockVector}

The \lstinline!template<class B> VariableBlockVector<B>! class
can be used to construct a vector space having a two-level
block structure of the form
$V=B^{n_1}\times B^{n_2}\times\ldots \times B^{n_m}$, i.e. it consists
of $m$ blocks $i=1,\ldots,m$ and each block in turn consists of $n_i$ blocks
given by the type \lstinline!B!. In principle this structure could be
built also with the previous classes but the implementation here is
more efficient. It allocates memory in one big array for all
components and for certain operations it is more efficient to
interpret the vector space as $V=B^{\sum_{i=1}^{m} n_i}$.

\subsection{Vectors are containers}

Vectors are containers over the base type \lstinline!K! or
\lstinline!B! in the sense of the Standard Template Library. Random
access is provided via \lstinline!operator[](int i)! where the indices
are in the range $0,\ldots,n-1$ with the number of blocks $n$ given by
the \lstinline!N! method. Here is a code fragment for illustration:
\begin{lstlisting}{}
Dune::BlockVector<Dune::FieldVector<std::complex<double>,2> > v(20);
v[1] = 3.14;
v[3][0] = 2.56;
v[3][1] = std::complex<double>(1,-1);
\end{lstlisting}
Note how one \lstinline!operator[]()! is used for each level of block
recursion. 

Sequential access to container elements is provided via
iterators. Here is a generic function accessing all the elements of a
vector:
\begin{lstlisting}{}
template<class V> void f (V& v)
{
  typedef typename V::Iterator iterator;
  for (iterator i=v.begin(); i!=v.end(); ++i)
    *i = i.index();
  
  typedef typename V::ConstIterator const_iterator;
  for (const_iterator i=v.begin(); i!=v.end(); ++i)
    std::cout << (*i).two_norm() << std::endl;
}
\end{lstlisting}
The \lstinline!Iterator! class provides read/write access while the
\lstinline!ConstIterator! provides read-only access. The type names
are accessed via the \lstinline!::!-operator from the scope of the
vector class.

A uniform naming scheme enables writing of generic algorithms. The
following types are provided in the scope of any vector class:

\subsection{Operations}

A full list of all members of a vector class is given in Figure
\ref{Fig:VectorMembers}. The norms are the same as defined for the
sparse BLAS standard \cite{BLASTForum}. The ``real'' variants avoid
the evaluation of a square root for each component in case of complex
vectors. The \lstinline!allocator_type! member type is explained below
in the section on memory management.

\begin{figure}
\begin{center}
\begin{tabular}{|l|l|l|}
\hline
\textbf{expression} & \textbf{return type} & \textbf{note}\\
\hline
\hline
\texttt{X::field\_type} & T & T is assignable\\
\hline
\texttt{X::block\_type} & T & T is assignable\\
\hline
\texttt{X::allocator\_type} & T & see mem.~mgt.\\
\hline
\texttt{X::blocklevel} & \texttt{int} & block levels inside\\
\hline
\texttt{X::Iterator} & T & read/write access\\
\hline
\texttt{X::ConstIterator} & T & read-only access\\
\hline
\texttt{X::operator=(field\_type\&)} & \texttt{X\&} & \\
\hline
\texttt{X::operator[](int)} & \texttt{field\_type\&} & \\
\hline
\texttt{X::operator[](int)} & \texttt{const field\_type\&} & \\
\hline
\texttt{X::operator+=(X\&)} & \texttt{X\&} & $x = x+y$\\
\hline
\texttt{X::operator-=(X\&)} & \texttt{X\&} & $x = x-y$\\
\hline
\texttt{X::operator*=(field\_type\&)} & \texttt{X\&} & $x =
\alpha x$\\
\hline
\texttt{X::operator/=(field\_type\&)} & \texttt{X\&} & $x =
\alpha^{-1} x$\\
\hline
\texttt{X::axpy(field\_type\&,X\&)} & \texttt{X\&} & $x = x+\alpha y$\\
\hline
\texttt{X::operator*(X\&)} & \texttt{field\_type} & $x\cdot y$\\
\hline
\texttt{X::one\_norm()} & \texttt{double} & $\sum_i\sqrt{Re(x_i)^2+Im(x_i)^2}$\\
\hline
\texttt{X::one\_norm\_real()} & \texttt{double} &$\sum_i(|Re(x_i)|+|Im(x_i)|)$\\
\hline
\texttt{X::two\_norm()} & \texttt{double} &$\sqrt{\sum_i(Re(x_i)^2+Im(x_i)^2)}$\\
\hline
\texttt{X::two\_norm2()} & \texttt{double} &$\sum_i (Re(x_i)^2+Im(x_i)^2)$\\
\hline
\texttt{X::infinity\_norm()} & \texttt{double} &$\max_i\sqrt{Re(x_i)^2+Im(x_i)^2}$\\
\hline
\texttt{X::infinity\_norm\_real()} & \texttt{double} &$\max_i(|Re(x_i)|+|Im(x_i)|)$\\
\hline
\texttt{X::N()} & \texttt{int} & number of blocks\\
\hline
\texttt{X::dim()} & \texttt{int} & dimension of space\\
\hline
\end{tabular}
\end{center}

\caption{Members of a vector class.}
\label{Fig:VectorMembers}
\end{figure}

\subsection{Object memory model and memory management}

The memory model for all ISTL objects is deep copy as in the Standard
Template Library and in contrast to the Matrix Template
Library. Therefore, references must be used to avoid excessive copying
of objects. On the other hand temporary vectors with appropriate
structure can be generated simply with the copy constructor.



\subsection{Vector creation}


\section{Matrices}

\subsection{Linear mappings}

\subsection{Matrix classes}

\subsection{Matrices are containers of containers}

\subsection{Operations}

\subsection{Matrix creation}

\section{Algorithms}

\subsection{Input/output}

\subsection{Simple iterative solvers}

\subsection{Sparse LU decomposition}

\section{Performance}

% bibtex bibliography
\bibliographystyle{plain}
\bibliography{istl.bib}

% some links
% http://www.netlib.org/blas/blast-forum/
% http://www.osl.iu.edu/research/mtl/
% http://www.osl.iu.edu/research/itl/ (based on MTL, mostly Krylov)
% http://netlib.org/utk/people/JackDongarra/la-sw.html
% the last one: FREELY AVAILABLE SOFTWARE FOR
% LINEAR ALGEBRA ON THE WEB (May 2004) by Jack Dongarra
% http://www.oonumerics.org/blitz/

\end{document}
